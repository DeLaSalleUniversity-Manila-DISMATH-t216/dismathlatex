\documentclass[compress,20]{beamer}
\mode<presentation>

\usetheme{Warsaw}
%\usefonttheme{default}
\usefonttheme{serif}
\hypersetup{pdfpagemode=FullScreen}

% define your own colours:
\definecolor{Red}{rgb}{1,0,0}
\definecolor{Blue}{rgb}{0,0,1}
\definecolor{Green}{rgb}{0,1,0}
\definecolor{magenta}{rgb}{1,0,.6}
\definecolor{lightblue}{rgb}{0,.5,1}
\definecolor{lightpurple}{rgb}{.6,.4,1}
\definecolor{gold}{rgb}{.6,.5,0}
\definecolor{orange}{rgb}{1,0.4,0}
\definecolor{hotpink}{rgb}{1,0,0.5}
\definecolor{newcolor2}{rgb}{.5,.3,.5}
\definecolor{newcolor}{rgb}{0,.3,1}
\definecolor{newcolor3}{rgb}{1,0,.35}
\definecolor{darkgreen1}{rgb}{0, .35, 0}
\definecolor{darkgreen2}{rgb}{0, .38, 0}
\definecolor{darkgreen}{rgb}{0, .6, 0}
\definecolor{darkred}{rgb}{.75,0,0}
\xdefinecolor{olive}{cmyk}{0.64,0,0.95,0.4}
\xdefinecolor{purpleish}{cmyk}{0.75,0.75,0,0}

\usecolortheme[named=darkgreen2]{structure} % Going Green

\useoutertheme{shadow}
\setbeamertemplate{background}[grid][step=0.5cm] %Masira ang sa taas
\usebeamertemplate*{logo}

% include packages
\usepackage{subfigure}
\usepackage{amstext}
\usepackage{latexsym}
\usepackage{epsfig}
\usepackage{graphicx}
\usepackage[all,knot]{xy}
\xyoption{arc}
\usepackage{url}
\usepackage{multimedia}
\usepackage{hyperref}
\usepackage{setspace}
%\usepackage{chess}
%\usepackage{graphpap}
%\usepackage{pstricks}
\usepackage{mathrsfs}
 

%My packages from mathbook
\usepackage{clock} %for clocks
\usepackage{ifsym} %  
\usepackage{epsdice} % dice
\usepackage{clock} % 
\usepackage{microtype}
\usepackage{skull} %create skull
\usepackage{cancel} %create diagonal bar (cancel)
\usepackage{stackrel} %create arc?
\usepackage{bbding} %create hands and cross
\usepackage{pifont} %create circled numbers
\usepackage{lipsum} % Just some sample text
\usepackage{amsmath,amssymb,amsthm} %math functions like align
\usepackage{fancybox} %For beautiful boxes
\usepackage{xspace} % Prints a trailing space in a smart way.
\usepackage{units}
\usepackage{geometry} % change paper size
\usepackage{multicol} % Small sections of multiple columns 
\usepackage{color}

%Animation
\usepackage{etex} % Solves ! No room for a new \dimen error
\usepackage{geometry}
\usepackage[T1]{fontenc}
\usepackage{lmodern}
\usepackage{tikz}
\usetikzlibrary{positioning,shadows,backgrounds}
\usepackage{animate}
\usepackage{calc}
\usepackage{color}
\usepackage{hyperref}
\usepackage{ifthen}



%My New Commands

\newcommand{\fig}[2]{
\begin{center}
\begin{figure}
\includegraphics[scale=#1]{figures/#2}
\end{figure}
\end{center}
}

%formula block
\newcommand{\f}[3]{
\begin{block}{\center  $ #1 $ }
\begin{center}
\begin{figure}
\includegraphics[scale=#2]{figures/#3}
\end{figure}
\end{center}
\end{block}
}

%question block
\newcommand{\q}[5]{
\begin{block}{#1 }
\begin{multicols}{2} \begin{enumerate}[]
\setlength\itemindent{0.69cm}
\item #2
\item #3
\item #4
\item #5
\end{enumerate}\end{multicols} 
\end{block}
}

\newcommand{\ans}[2]{\alert<#1>{\textbf<#1>{#2}}  \only<#1>{\textcolor {Red} \checkmark }} 


\newcommand{\dquote}[1]{\ding{125} \emph{#1} \ding{126}} %decorative quote

\newcommand{\formula}[1]{\vspace{-0.2cm}
\begin{center}
\shadowbox{#1}
\end{center}
\vspace{-0.2cm}
} % For formula


\newcommand{\point}[1]{\vspace{0.2cm} \HandCuffRight \, \smallcaps{#1}} %for pointing
\newcommand{\s}{\vspace{0.2cm}} % adds space
\newcommand{\ns}{\vspace{-0.5cm}}  % subtracts space
\newcommand{\g}[1]{\[\begin{gathered} #1 \end{gathered} \]}

%%% Steps %%%
\newcommand{\first}[1]{\ding{172} \smallcaps{#1}} 
\newcommand{\second}[1]{\ding{173}  \smallcaps{#1}} 
\newcommand{\third}[1]{\ding{174}  \smallcaps{#1}} 
\newcommand{\fourth}[1]{\ding{175}   \smallcaps{#1}} 


%%% EQUATIONS %%%
\newcommand{\one}{\quad \to \quad \text{\ding{182}}} 
\newcommand{\two}{\quad \to \quad \text{\ding{183}}} 
\newcommand{\three}{\quad \to \quad \text{\ding{184}}} 
\newcommand{\four}{\quad \to \quad \text{\ding{185}}} 
\newcommand{\five}{\quad \to \quad \text{\ding{186}}} 
\newcommand{\six}{\quad \to \quad \text{\ding{187}}} 
\newcommand{\seven}{\quad \to \quad \text{\ding{188}}} 
\newcommand{\eight}{\quad \to \quad \text{\ding{189}}} 


%My new environments

%New environment for choices in two columns %Modify to single column (4/14)
\newenvironment{choicescol}{\vspace{-0.12cm}
\begin{itemize}[]\setlength\itemindent{0.12cm} }
{\end{itemize} 
\vspace{0.2cm}
}

%New environment for choices in two columns
\newenvironment{choices}{\vspace{-0.4cm}
\begin{multicols}{2} \begin{itemize}[]\setlength\itemindent{0.69cm} }
{\end{itemize}\end{multicols}} 


 

\title{$\mathcal{DISMATH}$}
\subtitle{Discrete Mathematics and Its Applications \\
Welcome to DISMATH!
}
\author{Melvin Kong Cabatuan 
 }
\institute{{\large De La Salle University }  \\ 
Manila, Philippines \\

}

\date{\small September 2014 \ns }



\begin{document}
\begin{LARGE}
 



\frame{

\titlepage

\begin{center}
\begin{figure}
\includegraphics[scale=0.2]{figures/dlsulogo}
\end{figure}
\end{center}

}


\logo{\includegraphics[height=2cm]{figures/dlsulogo}}


\frame{ \frametitle{Self Introduction}
\ns
\fig{0.5}{me2}
\ns
 \noindent\underline{\textsc{\qquad Melvin K. Cabatuan, ECE}} 
 \begin{itemize}
\normalsize
\centering
\item Masters of Engineering, NAIST (Japan)
\item Thesis: Cognitive Radio (Wireless Communication)
\item IEEE Philippine Section Secretary (2012)
\item ECE Reviewer/Mentor (Since 2005)
\item 2nd Place, Nov. 2004 ECE Board Exam
\item Test Engineering Cadet, ON Semiconductors
\item DOST Academic Excellence Awardee 2004  
\item Mathematician of the Year 2003
\item DOST Scholar (1999-2004)
\item Panasonic Scholar, Japan (2007-2010)  
\end{itemize} 
\vfill
}


\frame{ 
\begin{center} 
\fig{0.8}{kyodai}
\vfill
\end{center} 
}

\frame{ 
\begin{center} 
\fig{0.7}{kenkyuu}
\vfill
\end{center} 
}


\frame{ 
\begin{center} 
\fig{0.5}{gundam}
\vfill
\end{center} 
}



\section{Introduction}

\frame{\tableofcontents}


\section{Course Contents}

\frame{\frametitle{Course Contents - Part I}
\setbeamercovered{transparent}
\begin{enumerate}
\Large
\item<1->  Logic, Sets, and Functions
\uncover<1->{\fig{0.3}{logic}}
\item<2->  Methods of Proof, Algorithms, Integers
\item<3->  Mathematical Reasoning, Induction, and Recursion
\only<3->{\fig{0.3}{math}}
\end{enumerate}
\vfill
}
 
\frame{\frametitle{Course Contents - Part II}
\setbeamercovered{transparent}
\begin{enumerate}
\Large
\item<1->  Relations
\uncover<1->{\fig{0.4}{relation}}
\item<2->  Graph Theory
\only<2->{\fig{0.3}{graph}}
\item<3->  Planar Graphs, Graph Coloration, and Trees
\only<3->{\fig{0.3}{graph2}}
\end{enumerate}
\vfill
}
 
\frame{\frametitle{Course Contents - Part III}
\setbeamercovered{transparent}

\begin{enumerate}
\Large

\item<1->  Counting Techniques and  Probability Theory
\only<1->{\fig{0.3}{aces} \quad \fig{0.3}{cointoss}}
\item<2->  Advance Counting Techniques 
\only<3->{\fig{0.5}{sierpinski}}

\end{enumerate}
\vfill
}

\frame{\frametitle{Course Contents - Part IV}
\setbeamercovered{transparent}
\begin{enumerate}
\Large

\item<1->  Modeling Computation, Finite State Machines and Automata 
\uncover<1->{\fig{0.25}{fsm}}
\item<2->  Algebraic Systems and Formal Languages
\only<2->{\fig{0.5}{language}}
\end{enumerate}
\vfill
}

\subsection{Evaluation Criteria}
\frame{\frametitle{Evaluation Criteria}
\setbeamercovered{transparent}
\begin{table}
  \centering
  \begin{tabular}{lc}
   Quiz Average: & 35\%  \\
    Final Exam: &  35\%   \\
    Project: & 25 \%   \\
    Teacher`s Evaluation: & 5\% \s  \\ 
       \hline \\ 
    Total: & 100\% \\
    PASSING GRADE: & 65\% \\
    \end{tabular}
\end{table}
\vfill
}

\subsection{Pre-requisite}
\frame{\frametitle{Pre-requisite}
\setbeamercovered{transparent}
\begin{enumerate}
\Large
\item<1->  ENGALG1 (Hard)
\item<2->  Mathematical Background (High-school mathematics should be enough if ...) 
\item<3->  A curious mind!
\end{enumerate}
\vfill
}


\subsection{References}
\frame{\frametitle{References}
\setbeamercovered{transparent}
\begin{enumerate}
\Large
\item<1->  Rosen, K.H. (2012). \emph{Discrete Mathematics and Its Applications} (7 ed.), New York, McGraw-Hill

\fig{0.152}{rosen}

\item<2->  Online Resources 

\end{enumerate}
\vfill
}


\section{Discrete Mathematics}



\frame{\frametitle{Discrete Mathematics}
\only<1>{
\begin{block}{Definition}
\dquote{Study of distinct \& countable objects.}
\end{block}
}
\only<2>{
\begin{block}{Purpose}
To provides the mathematical foundation for many computer engineering/science courses including data structures, algorithms, database theory, automata theory, formal languages, etc \ldots
\end{block}
}
\only<3>{
\begin{alertblock}{Insight}
\dquote{It excludes 'continuous mathematics' such as Calculus.}
\end{alertblock}
}
\fig{0.6}{discrete}
}

 

\subsection{Example}

\frame{\frametitle{Logic Example: Knights and Knaves}
\begin{block}{An island is inhabited only by knights and knaves. Knights always tell the truth, and knaves always lie.You meet two inhabitants: Mel and Vin. Determine what Mel and Vin is, if they say:}
Mel: \dquote{Vin is a knight} \\
Vin: \dquote{The two of us are opposite types}
\end{block}
\only<2>{
\begin{alertblock}{\center $\therefore$ Both Mel and Vin are Knaves!}
\end{alertblock}
}
}



\frame{\frametitle{Logic Example: Knights and Knaves}
\begin{block}{An island is inhabited only by knights and knaves. Knights always tell the truth, and knaves always lie.You meet two inhabitants: Mel and Vin. Determine what Mel and Vin is, if they say:}
Mel: \dquote{We are both knaves} \\
Vin: \dquote{\ldots}
\end{block}
\only<2>{
\begin{alertblock}{\center $\therefore$ Mel is a Knave and Vin is a Knight!}
\end{alertblock}
}
}





\frame{\frametitle{Example: Mathematical Reasoning/ Counting}
\begin{block}{A pyramid scheme promises participants payment, services, primarily for enrolling other people into the scheme or training them to take part, rather than supplying any real investment or sale of products.}
\fig{0.3}{pyramid}
\end{block}
} 
 

\frame{\frametitle{Example: Graph Theory/ Mapping}
\begin{block}{Facebook Map of the World.}
\fig{0.4}{facebook-map}
\end{block}
} 


\frame{\frametitle{Example: Trees}
\begin{block}{ Linux Directory.}
\fig{0.5}{linuxdir}
\end{block}
} 



\frame{\frametitle{Example: Modeling Computation}
\begin{block}{ Kasparov vs. Deep Blue.}
\fig{1}{deepblue}
\end{block}
} 


\frame{\frametitle{Key Insights}

\pause{
\begin{alertblock}{\Large \textsc{Discrete Mathematics} is the study of distinct \& countable objects.}
\end{alertblock}
}

\pause{
\begin{alertblock}{\Large It provides the mathematical foundation for many computer engineering/science courses including data structures, algorithms, database theory, automata theory, formal languages, etc \ldots}
\end{alertblock}
}

\pause{
\begin{alertblock}{\Large \dquote{It finds its application in our everyday lives.}}
\end{alertblock}
}
}


\frame{\frametitle{Shall we begin!}
\begin{center}
\dquote{Thank you for your attention}
\fig{0.45}{gundam}
\end{center} 
}


\end{LARGE}
\end{document}  
